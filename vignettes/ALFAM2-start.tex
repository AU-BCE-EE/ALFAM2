%% Next 2 lines needed for non-Sweave vignettes
%\VignetteEngine{knitr::knitr} 
%\VignetteIndexEntry{Getting started with the ALFAM2 package}
%\VignetteKeyword{ALFAM2}
%\VignetteKeyword{NH3}
%\VignetteKeyword{slurry}
%\VignetteKeyword{manure}
%\VignetteEncoding{UTF-8}

\documentclass{article}

%%\usepackage[version=3]{mhchem} %chemical formulas
\usepackage[colorlinks = true, urlcolor = blue]{hyperref} % Must be loaded as the last package



\title{Getting started with the ALFAM2 package}
\author{Sasha D. Hafner (\texttt{sasha@hafnerconsulting.com, sasha.hafner@eng.au.dk})}

\begin{document}

\maketitle

\section{Introduction}
The ALFAM2 project is on ammonia volatilization (emission) from field-applied manure, and includes two main products: a database with volatilization measurements and a model for estimating volatilization.
The model, which is described in detail in \cite{afmod2019}, is the focus of the ALFAM2 R package and this document.
The ALFAM2 package is an add-on package for R, which is an environment for statistical computing. 
With the model, it is possible to predict average volatilization rate and cumulative emission over time, as affected by application method, weather, and manure characteristics.
This document provides an introduction to the use of the model, and is aimed for users who are new to R.
Those with some knowledge of R can skip Section 2.

\subsection{Excel or R?}
The ALFAM2 model is available in an Excel spreadsheet in addition to the R package that is described in this document.
If you would just like to know cumulative emission for a few scenarios with constant conditions, the Excel model is a good choice.
But to work with many different scenarios, or when weather changes over time (e.g., wind or rain), or if you are interested in emission dynamics and not just final cumulative emission, you should use the R package.
You can use the ALFAM2 package without much knowledge of R.
If you are not currently an R user, but you plan on using the ALFAM2 model extensively, it is worthwhile to learn a little bit about R and use the ALFAM2 package, instead of the less efficient Excel spreadsheet model.

\section{Some basics for new R users}
The information given in this section should be enough for new R users to install the package and learn enough about R to start using the model (albeit with a lack of understanding about some of the code) as described in Section 3.
For a better understanding, check out the sources mentioned below.

\subsection{Getting started with R}
To use R it must be installed on your computer.
You can download R and find installation instructions from here: \url{https://www.r-project.org/}.
And while not required, it is convenient to have a good script editor.
The RStudio IDE (integrated development environment) is a good (and very popular) choice.
It can be downloaded from here: \url{https://rstudio.com/products/rstudio/download/}.

To use the ALFAM2 package, you will need to install the package, and then call up the function. 
In R, you will need to be able to install and load packages, call functions, and, ideally, create data frames and export data.
For information on these tasks and more, there are many free resources online. 
I recommend this book I use for a course on R: \url{https://www.researchgate.net/publication/325170649_An_Introduction_to_R_for_Beginners}.
CRAN provides various manuals, including a good introduction: \url{https://cran.r-project.org/} (select "Manuals" at the lower left).
RStudio also provides various materials for learning R, although the focus is skewed toward packages developed by RStudio employees: \url{https://education.rstudio.com/learn/}.
Alternatively, the instructions given below may be sufficient.

\subsection{Installing the ALFAM2 package}
The ALFAM2 package is available from a GitHub repository: \url{https://github.com/sashahafner/ALFAM2}.
Installation of packages from GitHub requires a package called devtools.
You can run the code below to install devtools and ALFAM2.

First, install devtools from CRAN.

\begin{knitrout}
\definecolor{shadecolor}{rgb}{0.969, 0.969, 0.969}\color{fgcolor}\begin{kframe}
\begin{alltt}
\hlkwd{install.packages}\hlstd{(}\hlstr{"devtools"}\hlstd{)}
\end{alltt}
\end{kframe}
\end{knitrout}

Then load the package,

\begin{knitrout}
\definecolor{shadecolor}{rgb}{0.969, 0.969, 0.969}\color{fgcolor}\begin{kframe}
\begin{alltt}
\hlkwd{library}\hlstd{(devtools)}
\end{alltt}
\end{kframe}
\end{knitrout}

and install the ALFAM2 package from GitHub.\footnote{
  Some additional notes.
  You need the \texttt{build\_vignettes = TRUE} bit to install this vignette that you are now reading (and any others that may be added in the future).
  To get the latest version of the package (possible bugs, incomplete testing, and all), add the argument \texttt{ref = "dev"}.
}

\begin{knitrout}
\definecolor{shadecolor}{rgb}{0.969, 0.969, 0.969}\color{fgcolor}\begin{kframe}
\begin{alltt}
\hlkwd{install_github}\hlstd{(}\hlstr{"sashahafner/ALFAM2"}\hlstd{,} \hlkwc{build_vignettes} \hlstd{=} \hlnum{TRUE}\hlstd{)}
\end{alltt}
\end{kframe}
\end{knitrout}

Alternatively, to avoid loading devtools, use this syntax.

\begin{knitrout}
\definecolor{shadecolor}{rgb}{0.969, 0.969, 0.969}\color{fgcolor}\begin{kframe}
\begin{alltt}
\hlstd{devtools}\hlopt{::}\hlkwd{install_github}\hlstd{(}\hlstr{"sashahafner/ALFAM2"}\hlstd{,} \hlkwc{build_vignettes} \hlstd{=} \hlnum{TRUE}\hlstd{)}
\end{alltt}
\end{kframe}
\end{knitrout}

These steps only need to be carried out once.

Finally, every time you open R to use the ALFAM2 package, it must be loaded.

\begin{knitrout}
\definecolor{shadecolor}{rgb}{0.969, 0.969, 0.969}\color{fgcolor}\begin{kframe}
\begin{alltt}
\hlkwd{library}\hlstd{(ALFAM2)}
\end{alltt}
\end{kframe}
\end{knitrout}

You can open this vignette with the following call.

\begin{knitrout}
\definecolor{shadecolor}{rgb}{0.969, 0.969, 0.969}\color{fgcolor}\begin{kframe}
\begin{alltt}
\hlkwd{vignette}\hlstd{(}\hlstr{"ALFAM2-start"}\hlstd{)}
\end{alltt}
\end{kframe}
\end{knitrout}


\section{The \texttt{ALFAM2mod()} function}
The ALFAM2 package includes a single function that is an implementation of the ALFAM2 model: \texttt{ALFAM2mod()}
After an explanation of the function, its use is shown in a few examples.

\subsection{Overview of the function}
The \texttt{ALFAM2()} function can be used for predicting average volatilization rate and cumulative emission over time.
The function has several arguments, as shown below.

\begin{knitrout}
\definecolor{shadecolor}{rgb}{0.969, 0.969, 0.969}\color{fgcolor}\begin{kframe}
\begin{alltt}
\hlkwd{args}\hlstd{(ALFAM2mod)}
\end{alltt}
\begin{verbatim}
## function (dat, pars = ALFAM2::ALFAM2pars02, app.name = "TAN.app", 
##     time.name = "ct", time.incorp = NULL, group = NULL, center = TRUE, 
##     cmns = c(app.rate = 40, man.dm = 6, man.tan = 1.2, man.ph = 7.5, 
##         air.temp = 13, wind.2m = 2.7, crop.z = 10), check.NA = TRUE, 
##     pass.col = NULL, incorp.names = c("incorp", "deep", "shallow"), 
##     add.incorp.rows = FALSE, warn = TRUE, prep = FALSE, parallel = FALSE, 
##     n.cpus = 1, ...) 
## NULL
\end{verbatim}
\end{kframe}
\end{knitrout}

You can find more details on the arguments (as well as examples) in the help file.
As with any R function, you can open the file with \texttt{?}:

\begin{knitrout}
\definecolor{shadecolor}{rgb}{0.969, 0.969, 0.969}\color{fgcolor}\begin{kframe}
\begin{alltt}
\hlopt{?}\hlstd{ALFAM2mod}
\end{alltt}
\end{kframe}
\end{knitrout}

But the most important arguments are described here.
Most arguments have default values, and the only one that is required to make predictions is the \texttt{dat} argument, which is a data frame containing some input data, i.e., values of predictor variables over time.
The \texttt{dat} data frame can contain any number of rows (each representing a time interval), but must contain a column with cumulative time in hours, and the name of this column is indicated with \texttt{time.name}.
Typically the data frame will have predictor variables as well, for example, manure dry matter, application method, air temperature, or wind speed.
The name of the predictor columns are used to link predictor variables to model parameters, which are set by the \texttt{pars} argument. 
Usually the default values, based on the measurements in the ALFAM2 database, should be used.
Predictor variables and their default names are given in Table 1 below.

\begin{table}[]
  \caption{Default predictor variables that can be used with \texttt{ALFAM2mod()}, as given in the \texttt{ALFAM2pars02} or \texttt{ALFAM2pars01} objects.}
\begin{tabular}{lllll}
\hline
Variable name & Description             & Units          & Notes             &  \\
\hline
  \texttt{int}           & Intercept terms         & None           &                   &  \\
  \texttt{app.mthd.os}  & Open slot application   & None (logical) & Binary variable   &  \\
  \texttt{app.mthd.cs}  & Closed slot application   & None (logical) & Binary variable   &  \\
  \texttt{app.mthd.bc}  & Broadcast application   & None (logical) & Binary variable   &  \\
  \texttt{app.mthd.ts}  & Trailing shoe application   & None (logical) & Binary variable   &  \\
  \texttt{app.rate}      & Manure application rate & t/ha           &                   &  \\
  \texttt{app.rate.ni}   & Manure app. rate (excluding ("no") injection) & t/ha           &                   &  \\
  \texttt{man.dm}        & Manure dry matter       & \%             &                   &  \\
  \texttt{man.ph}        & Manure pH               & pH units       & For acidification &  \\
  \texttt{man.source.pig}        & Pig manure    & None (logical)  & Binary variable    &  \\
\texttt{incorp.deep}    & Deep incorporation      & None (logical) & Binary variable   &  \\
\texttt{incorp.shallow} & Shallow incorporation   & None (logical) & Binary variable   &  \\
  \texttt{air.temp}      & Air tempreature         & $^\circ$C      &                   &  \\
  \texttt{wind.2m}       & Wind speed (at 2 m)     & m/s            &                   &  \\
  \texttt{rain.rate}     & Rainfall rate           & mm/h           &                   &  \\
  \texttt{rain.cum}      & Cumulative rain         & mm             &                   &  \\
  \texttt{cereal.hght}      & Cereal height         & cm             &                   &  \\
\hline
\end{tabular}
\end{table}

Default model parameters and numeric values in the \texttt{ALFAM2pars02} object (``Set 2'') should generally be used. 
For information on how these values were calculated, see the report on calculation of Danish emission factors \cite{efreport2021}.
(An earlier version (``Set 1'') are available in \texttt{ALFAM2pars01}. Derivation of these is described in the 2019 paper \cite{afmod2019}.) 
Comparing the contents of \texttt{ALFAM2pars02} to the variable names given in Table 1, you can see an additional letter and number added to the end of the parameters. 

\begin{knitrout}
\definecolor{shadecolor}{rgb}{0.969, 0.969, 0.969}\color{fgcolor}\begin{kframe}
\begin{alltt}
\hlstd{ALFAM2pars02}
\end{alltt}
\begin{verbatim}
##            int.f0    app.mthd.os.f0    app.rate.ni.f0 
##       -0.60568338       -1.74351499       -0.01114900 
##         man.dm.f0 man.source.pig.f0    app.mthd.cs.f0 
##        0.39967070       -0.59202858       -7.63373787 
##            int.r1    app.mthd.bc.r1         man.dm.r1 
##       -0.93921516        0.79352480       -0.13988189 
##       air.temp.r1        wind.2m.r1    app.mthd.ts.r1 
##        0.07354268        0.15026720       -0.45907135 
## ts.cereal.hght.r1         man.ph.r1            int.r2 
##       -0.24471238        0.66500000       -1.79918546 
##      rain.rate.r2            int.r3    app.mthd.bc.r3 
##        0.39402156       -3.22841225        0.56153956 
##    app.mthd.cs.r3         man.ph.r3 incorp.shallow.f4 
##       -0.66647417        0.23800000       -0.96496655 
## incorp.shallow.r3    incorp.deep.f4    incorp.deep.r3 
##       -0.58052689       -3.69494954       -1.26569562
\end{verbatim}
\end{kframe}
\end{knitrout}

These numbers indicate a primary parameter.
So, for example, the (secondary) parameter \texttt{wind.2m.r1}, which is 0.15 s/m by default, is used in the calculation of the primary parameter $r_1$.
The most important message here is a simple one: names for predictor variables can be taken from the names given in the default \texttt{pars} argument value, but be sure to omit the last three characters (a ".", a number, and a letter).

By design, any time a predictor variable is omitted when \texttt{ALFAM2mod()} is called, the reference level or value is assumed for that variable.\footnote{
One exception is \texttt{app.rate.ni}.
}
The scenario with reference levels for all predictors is the default scenario, and is the one given in the first row of Table 4 in \cite{afmod2019}.
Predictor values for the default scenario can be found in the \texttt(cmns) argument (for centering means, see help file).
The default application method is trailing hose.
The \texttt{cmns} argument is used for centering predictor variables, and this approach facilities the behavior described above.

\subsection{Cumulative emission for a single scenario}
In this example, let's assume we are interested in manure application by broadcast when manure had 8\% dry matter (DM), total TAN application is 50 kg/ha, wind is 3 m/s, and air temperature is 20$^\circ$C.

First we need to create a data frame with the input data.

\begin{knitrout}
\definecolor{shadecolor}{rgb}{0.969, 0.969, 0.969}\color{fgcolor}\begin{kframe}
\begin{alltt}
\hlstd{dat1} \hlkwb{<-} \hlkwd{data.frame}\hlstd{(}\hlkwc{ctime} \hlstd{=} \hlnum{72}\hlstd{,} \hlkwc{TAN.app} \hlstd{=} \hlnum{50}\hlstd{,} \hlkwc{man.dm} \hlstd{=} \hlnum{8}\hlstd{,}
                   \hlkwc{air.temp} \hlstd{=} \hlnum{20}\hlstd{,} \hlkwc{wind.2m} \hlstd{=} \hlnum{3}\hlstd{,}
                   \hlkwc{app.mthd.bc} \hlstd{=} \hlnum{TRUE}\hlstd{)}
\hlstd{dat1}
\end{alltt}
\begin{verbatim}
##   ctime TAN.app man.dm air.temp wind.2m app.mthd.bc
## 1    72      50      8       20       3        TRUE
\end{verbatim}
\end{kframe}
\end{knitrout}

Our predictor variable values are in the columns \texttt{man.dm} and the following ones.
The names for the predictor variables must match those names used in the model parameters, which can be seen by checking the parameter object contents (see just above).

Time, in hours after application, is given in the column named \texttt{ctime} here, for cumulative time (although any name can be used).

And now we can call the model function, using default values for most other arguments.
We can predict cumulative emission after 3 days (72 hours) with the following call.

\begin{knitrout}
\definecolor{shadecolor}{rgb}{0.969, 0.969, 0.969}\color{fgcolor}\begin{kframe}
\begin{alltt}
\hlstd{pred1} \hlkwb{<-} \hlkwd{ALFAM2mod}\hlstd{(dat1,} \hlkwc{app.name} \hlstd{=} \hlstr{'TAN.app'}\hlstd{,} \hlkwc{time.name} \hlstd{=} \hlstr{'ctime'}\hlstd{)}
\end{alltt}


{\ttfamily\noindent\itshape\color{messagecolor}{\#\# Default parameters (Set 2) are being used.}}

{\ttfamily\noindent\color{warningcolor}{\#\# Warning in ALFAM2mod(dat1, app.name = "{}TAN.app"{}, time.name = "{}ctime"{}): Running with 10 parameters. Dropped 14 with no match.\\\#\# These secondary parameters have been dropped:\\\#\# \ \ app.mthd.os.f0\\\#\# \ \ app.rate.ni.f0\\\#\# \ \ man.source.pig.f0\\\#\# \ \ app.mthd.cs.f0\\\#\# \ \ app.mthd.ts.r1\\\#\# \ \ ts.cereal.hght.r1\\\#\# \ \ man.ph.r1\\\#\# \ \ rain.rate.r2\\\#\# \ \ app.mthd.cs.r3\\\#\# \ \ man.ph.r3\\\#\# \ \ incorp.shallow.f4\\\#\# \ \ incorp.shallow.r3\\\#\# \ \ incorp.deep.f4\\\#\# \ \ incorp.deep.r3\\\#\# \\\#\# These secondary parameters are being used:\\\#\# \ \ int.f0\\\#\# \ \ man.dm.f0\\\#\# \ \ int.r1\\\#\# \ \ app.mthd.bc.r1\\\#\# \ \ man.dm.r1\\\#\# \ \ air.temp.r1\\\#\# \ \ wind.2m.r1\\\#\# \ \ int.r2\\\#\# \ \ int.r3\\\#\# \ \ app.mthd.bc.r3}}\end{kframe}
\end{knitrout}

The warning message just tells us that the call included some parameters with no associated predictor variables in our data frame given in the \texttt{dat} argument.
This is discussed more below.
We will turn off the warning in the examples below.

Let's look at the predictions.

\begin{knitrout}
\definecolor{shadecolor}{rgb}{0.969, 0.969, 0.969}\color{fgcolor}\begin{kframe}
\begin{alltt}
\hlstd{pred1}
\end{alltt}
\begin{verbatim}
##   ct dt        f0       r1         r2          r3 f4
## 1 72 72 0.5482638 1.362777 0.01587869 0.002153413  1
##              f        s         j        e    e.int        er
## 1 2.130583e-42 19.61361 0.4220332 30.38639 30.38639 0.6077278
\end{verbatim}
\end{kframe}
\end{knitrout}

The most interesting columns here are called \texttt{e}, which has cumulative emission in the same units as TAN application, and \texttt{er}, which has relative cumulative emission, as a fraction of applied TAN.
So in this example, 48\% of applied TAN is predicted to be lost by volatilization.

The warning message above is related to an important point: Any excluded predictors are effectively assumed to be at their reference levels.

\subsection{Adding incorporation}
To include incorporation, we need to add a couple columns to our data frame.
First let's make a new data frame for the example.

\begin{knitrout}
\definecolor{shadecolor}{rgb}{0.969, 0.969, 0.969}\color{fgcolor}\begin{kframe}
\begin{alltt}
\hlstd{dat2} \hlkwb{<-} \hlstd{dat1}
\end{alltt}
\end{kframe}
\end{knitrout}

And add the two new columns.
Here we are specifying that deep incorporation happens after 0.5 hours.

\begin{knitrout}
\definecolor{shadecolor}{rgb}{0.969, 0.969, 0.969}\color{fgcolor}\begin{kframe}
\begin{alltt}
\hlstd{dat2}\hlopt{$}\hlstd{incorp.deep} \hlkwb{<-} \hlnum{TRUE}
\hlstd{dat2}\hlopt{$}\hlstd{t.incorp} \hlkwb{<-} \hlnum{0.5}
\hlstd{dat2}
\end{alltt}
\begin{verbatim}
##   ctime TAN.app man.dm air.temp wind.2m app.mthd.bc incorp.deep
## 1    72      50      8       20       3        TRUE        TRUE
##   t.incorp
## 1      0.5
\end{verbatim}
\end{kframe}
\end{knitrout}

\begin{knitrout}
\definecolor{shadecolor}{rgb}{0.969, 0.969, 0.969}\color{fgcolor}\begin{kframe}
\begin{alltt}
\hlstd{pred2} \hlkwb{<-} \hlkwd{ALFAM2mod}\hlstd{(dat2,} \hlkwc{app.name} \hlstd{=} \hlstr{"TAN.app"}\hlstd{,} \hlkwc{time.name} \hlstd{=} \hlstr{"ctime"}\hlstd{,}
                   \hlkwc{time.incorp} \hlstd{=} \hlstr{"t.incorp"}\hlstd{,} \hlkwc{warn} \hlstd{=} \hlnum{FALSE}\hlstd{)}
\end{alltt}


{\ttfamily\noindent\bfseries\color{errorcolor}{\#\# Error in rowSums(dat[, inc.ex]): 'x' must be an array of at least two dimensions}}\begin{alltt}
\hlstd{pred2}
\end{alltt}


{\ttfamily\noindent\bfseries\color{errorcolor}{\#\# Error in eval(expr, envir, enclos): object 'pred2' not found}}\end{kframe}
\end{knitrout}





































